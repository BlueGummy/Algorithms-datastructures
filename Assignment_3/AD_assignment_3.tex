\documentclass[12pt]{article}
\usepackage{amsmath} % flere matematikkommandoer
\usepackage[utf8]{inputenc} % æøå
\usepackage[T1]{fontenc} % mere æøå
\usepackage[danish]{babel} % orddeling
\usepackage{verbatim} % så man kan skrive ren tekst
\usepackage[all]{xy} % den sidste (avancerede) formel i dokumentet
\usepackage{amsmath}
\usepackage{amsfonts}
\usepackage{amssymb}
\usepackage{graphicx}
\usepackage{fancyhdr}
\usepackage{moreverb}
\usepackage[a4paper, hmargin={2.8cm, 2.8cm}, vmargin={2.5cm, 2.5cm}]{geometry}


\title{Algorithms and Datastructures assignment 3}
\author{Thomas Broby Nielsen (xlq119)\\ Tobias Overgaard (vqg954)\\ Christian Buchter (zvc154)}

\begin{document}
\maketitle

\tableofcontents

\pagebreak
\section{Task 1}
\begin{verbatim}
possible(p, b, n)
1 r=0
2 for m=1 to p.length
3      if m=p.length
4         if r >= 0 return true
5         else return false
6      if b[m] > n+((m+1)-m)
7         r=r+(b[m]-(n+((m+1)-m)))
8      if b[m]< n
9         r=r-(n-(b[m]+((m+1)-m)))
10      else r=r+0

\end{verbatim}
\newpage
\section{task 2}
Problemet har optimal substruktur da i kun skal finde possiblity, så er hvilken som helst løsning hvor hver bar får  $\bar{b}$ øl en optimal løsning.\\\\
Vores algortime har Greedy choice Property.\\
Det grådige valg går ud på at den første bar skal have præcist  $\bar{b}$ øl. Da distancen mellem barerne er linær koster det ikke ekstra øl at stoppe på alle barerne. Ved den første bar er der to valgmuligheder. Hvis $b_1$ <  $\bar{b}$ så skal der sendes  $\bar{b}-$$b_1$+($p_2-p_1$) øl til bar 1 fra bar 2. Hvis $b_1 >  \bar{b}$ så kan den optimale løsning ikke sende mere end $b_1- \bar{b}$.

\newpage
\section{Task 3}
\begin{verbatim}
maxBeer(p, b)
1 l=0
2 h=max(b)
2 while l < h
3      m = celing((l+h)/2)
4       if possible(b,p,m)
5          l=m
6       else h=m-1
\end{verbatim}
Denne algoritm vil bruge =(log B) iterationer af possible, hvilket giver køretiden: O(n lg B) Algoritmen virker fordi at hvis vi kan få x øl i hver bar, kan vi også få x' < x øl ved først at få  øl, og så bare kassere øl ved hver bar.

\end{document}