\documentclass[12pt]{article}
\usepackage{amsmath} % flere matematikkommandoer
\usepackage[utf8]{inputenc} % æøå
\usepackage[T1]{fontenc} % mere æøå
\usepackage[danish]{babel} % orddeling
\usepackage{verbatim} % så man kan skrive ren tekst
\usepackage[all]{xy} % den sidste (avancerede) formel i dokumentet
\usepackage{amsmath}
\usepackage{amsfonts}
\usepackage{amssymb}
\usepackage{graphicx}
\usepackage{fancyhdr}
\usepackage{moreverb}
\usepackage[a4paper, hmargin={2.8cm, 2.8cm}, vmargin={2.5cm, 2.5cm}]{geometry}


\title{Algorithms and Datastructures assignment 2}
\author{Thomas Broby Nielsen (xlq119)\\ Tobias Overgaard (vqg954)\\ Christian Buchter (zvc154)}

\begin{document}
\maketitle

\tableofcontents

\pagebreak
\section{Task 1}
$$
|N(c,i)| = \left\{ \begin{array}{rl}
0 &\mbox{ if $c \leq 0$} \\
0 &\mbox{ if $i=0$} \\
1+N(c,i-1) &\mbox{ if $p_i=c$}\\
N(c-p_i,i-1)+N(c,i-1) &\mbox{otherwise}
\end{array} \right.
$$

\section{task 2}
In order to prove the correctness of our formula, for calculating the amount of unique combinations of unique beers you can buy with a specified amount of money, we will show that our formula devides a problem into overlapping sub-problems, and solves those subproblems correctly. 
First we have to establish the correctness of our base cases.
To Illustrate this, if the algorithm starts out with $c \leq 0$ or if it starts out with $i = 0$ where $i \in \mathbb{Z}$, the algorithm will return 0. If at some point $c = 0$ occurs and c started out as $0 \leq c$  our algorithm will returns 1. If the same scenario happens but c goes bellow 0, $c < 0$, our algorithm will return 0. If both $c \geq  0$ and and $i > 1$ then our algorithm will make a tree like structure and divide the problem into  overlapping sub-problems. These sub-problems will then be solved by the algorithm in accordance with the base cases.   

 

\newpage
\section{Task 3}
Bellow we have written the pseudo-code for MemoizedBeerComp using the DP-method\\
\begin{verbatim}
MemoizedBeerComp(c, i, beerList, holderMatrix)
1 if c <= 0 or i==0
2    return 0
3 if array(c, i) > -1
4    return holderMatrix(c, i)
5 if c - beerList[i] == 0
6     holderMatrix(c,i) = 1 + MemoizedBeerComp(c, i - 1)
7     return holderMatrix(c,i)
8   else
9        holderMatrix(c,i) =
            MemoizedBeerComp(c, i - 1, beerList, holderMatrix) +
            MemoizedBeerComp(c - beerList[i], i - 1, beerLis, holderMatrix)
10       return holderMatrix(c, i)
\end{verbatim}
Since we weren't able to calculate the running time for our memoizedBeerComp for task 4, with our top-down implementation of memoizedBeerComp, we decided to also implement our original formula as a bottom-up method instead, as seen below.
\begin{verbatim}
Bottom-Up-Beers (P,c) = 
1   let R[1,2, ... ,c][0,1,2,...,P.length] be a new Matrix
2   for j=1 to c{
3      R[j][0] = 0;
4      for k = 1 to P.length {
5         q = R[j][k-1];
6         if (P[k] == j){ 
7             q+=1}   
8         if (P[k] < j) {
9            q += R[j-P[k]][k-1];
10         }
11     R[j][k] = q;	
12     }
13  }
14	return R[c][P.length];
\end{verbatim}
NOTE: we later figured out how to calculate the running time for the top-down method implementation after talking with a TA, but we decided to keep both of the implementations.
\section{Task 4}
For this task we went ahead and analysed the bottom-up implementation, Bottom-Up-Beers. From the nested for-loop in our implementation, first seen at line 2 and then again inside the first for-loop at line 4, we can see that our algorithm first checks through all possible combinations of $k \geq p.lenght$ for some $j \geq c$. From this it is we can see that the upper and lower bound for our running time is $O(cn)$ and $\omega(cn)$. For clarity n is equal to p.length with is the amount of individual beers there are avaliable and c is the amount of money there is to spend. Since our upper and lower bound is $c \cdot n$, our running time is therefore $T(n) = \theta(cn)$.\\

Since our implementation stores every single combination of $c \cdot i$, we can visualize that our implementation stores it's asnwers as a matrix, where the amount of colums and rows of the matrix are determined by the size of "c" and "i". Therefore since our matrix has to be stored for the algorithm to access it our memory usage is ||R|| = cn.\\

We prove the correctness by using the loop invariants:
\begin{itemize}
\item At the start each iteration of the inner loop the value of R[$j$][$k-1$] is the number of combinations of the p[$1..k-1$] beers achievable with $j$ kr.
\item At the start of each iteration of the outer loop the value of R[$j-1$][$n$], where $n$ is P.length, is the number of combinations of all the beers achievable with $j-1$ kr.

\end{itemize}

\textbf{Initialisation}\\
Before the first inner loop the value of R[j][0] is 0, which is the number of combinations of 0 beers achievable with j kr.\\

Before the first outer loop the value of R[0][n] is 0, which is the number of combinations of all the beers achievable with 0 kr.\\

\textbf{Maintenance}\\

If R[$j$][$k-1$] is true before the $k'th$ inner loop R[$j$][$k$] will be true after it since if $k = j$ we add that as a standalone combination to R[$j$][$k-1$], and otherwise if $j>$ P[$k$] we add the combinations from R[$j-P[k]$][$k-1$] which gives us all the combinations of R[$j$][$k$] as long as earlier iterations were Initialised correctly.\\

If R[$j-1$][$n$] is true before the $j'th$ iteration of the inner loop R[$j$][$n$] will be true after it since if the loop invariant for the inner loop holds after termination the $n'th$ iteration of the inner loop R[$j$][$n$] will contain the number of combinations of all the beers achievable with $j$ kr.\\

\textbf{Termination}\\

The condition causing the inner loop to terminate is $k > n$. because each loop iteration increases k by one and because $n \neq \infty$ the loop must terminate. Since Maintenance of the inner loop gives us that R[$j$][$n$] will be true after the $n'th$ iteration we have that the loop will terminate and give us the loop invariant of the outer loop after the $n'th$ iteration.\\

The condition causing the outer loop to terminate is $j > c$. because each loop iteration increases j by one and because $c \neq \infty$ the loop must terminate. Since Maintenance of the outer loop gives us that R[$c$][$n$] will be true after the $c'th$ iteration we have that the loop will terminate after the $c'th$  iteration and give us the total number of combinations of the $n$ beers achievable with $c$ kr.\\

\bibliographystyle{plain}
\bibliography{references}
\end{document}