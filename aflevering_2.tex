\documentclass[12pt]{article}
\usepackage{amsmath} % flere matematikkommandoer
\usepackage[utf8]{inputenc} % æøå
\usepackage[T1]{fontenc} % mere æøå
\usepackage[danish]{babel} % orddeling
\usepackage{verbatim} % så man kan skrive ren tekst
\usepackage[all]{xy} % den sidste (avancerede) formel i dokumentet
\usepackage{amsmath}
\usepackage{amsfonts}
\usepackage{amssymb}
\usepackage{graphicx}
\usepackage{fancyhdr}
\usepackage{moreverb}


\title{Algorithms and Datastructures assignment 1}
\author{Thomas Broby Nielsen (xlq119)\\ Tobias Overgaard (vqg954)\\ Christian Buchter (zvc154)}

\begin{document}
\maketitle

\tableofcontents

\pagebreak
\section{Task 1}
$$
|N(c,i)| = \left\{ \begin{array}{rl}
0 &\mbox{ if $c \leq 0$} \\
0 &\mbox{ if $i=0$} \\
1+N(c,i-1) &\mbox{ if $p_i=c$}\\
N(c-p_i,i-1)+N(c,i-1) &\mbox{otherwise}
\end{array} \right.
$$

\newpage
\section{task 2}

\newpage
\section{Task 3}
Bellow we have written the pseudo-code for MemoizedBeerComp using memoization DP\\

\begin{verbatim}
MemoizedBeerComp(c, i, P, R)
1 if c <= 0 or i==0
2    return 0
3 if R(c,i) > -1
4    return R(c,i)
5 if c-P(i)==0
6     R(c,i)=1+ MemoizedBeerComp(c,i-1)
7     return R(c,i)
8  else
9     R(c,i)=MemoizedBeerComp(c,i-1, P, R)+MemoizedBeerComp(c-P[i], i-1, P, R)
10   return R(c,i)
\end{verbatim}
\section{Task 4}



\bibliographystyle{plain}
\bibliography{references}
\end{document}