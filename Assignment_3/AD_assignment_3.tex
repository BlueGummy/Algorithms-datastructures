\documentclass[12pt]{article}
\usepackage{amsmath} % flere matematikkommandoer
\usepackage[utf8]{inputenc} % æøå
\usepackage[T1]{fontenc} % mere æøå
\usepackage[danish]{babel} % orddeling
\usepackage{verbatim} % så man kan skrive ren tekst
\usepackage[all]{xy} % den sidste (avancerede) formel i dokumentet
\usepackage{amsmath}
\usepackage{amsfonts}
\usepackage{amssymb}
\usepackage{graphicx}
\usepackage{fancyhdr}
\usepackage{moreverb}
\usepackage[a4paper, hmargin={2.8cm, 2.8cm}, vmargin={2.5cm, 2.5cm}]{geometry}


\title{Algorithms and Datastructures assignment 3}
\author{Thomas Broby Nielsen (xlq119)\\ Tobias Overgaard (vqg954)\\ Christian Buchter (zvc154)}

\begin{document}
\maketitle

\tableofcontents

\pagebreak
\section{Task 1}
\begin{verbatim}
possible(p, b, n)
1 r=0
2 for m=1 to p.length
3      if m=p.length
4         if b[m] > n+(m-(m-1))
5            r=r+(b[m]-(n+(m-(m-1))))
6         if b[m] < n
7            r=r-(n-(b[m]+(m-(m-1))))
8         else r=r+0
9      if b[m] > n+((m+1)-m)
10         r=r+(b[m]-(n+((m+1)-m)))
11      if b[m]< n
12         r=r-(n-(b[m]+((m+1)-m)))
13      else r=r+0
14 return r
\end{verbatim}
\section{task 2}

\newpage
\section{Task 3}


\bibliographystyle{plain}
\bibliography{references}
\end{document}